\documentclass[11pt,a4paper]{moderncv}

% moderncv themes
%\moderncvtheme[blue]{casual}                 % optional argument are 'blue' (default), 'orange', 'red', 'green', 'grey' and 'roman' (for roman fonts, instead of sans serif fonts)
\moderncvtheme[blue]{classic}                % idem
\usepackage{xunicode, xltxtra}
\XeTeXlinebreaklocale "zh"
\widowpenalty=10000

%\setmainfont[Mapping=tex-text]{文泉驿正黑}
%\setmainfont[BoldFont=黑体]{宋体}

% character encoding
%\usepackage[utf8]{inputenc}                   % replace by the encoding you are using
\usepackage{CJKutf8}

% adjust the page margins
\usepackage[scale=0.8]{geometry}
\recomputelengths                             % required when changes are made to page layout lengths
\setmainfont[Mapping=tex-text]{STFangsong}
\setsansfont[Mapping=tex-text]{STFangsong}
\CJKtilde

% personal data
\firstname{朱佳顺}
\familyname{}
\title{简历}               % optional, remove the line if not wanted

\mobile{+86 13764325590}                    % optional, remove the line if not wanted
\email{zhujiashun2010@gmail.com}                      % optional, remove the line if not wanted
%% \quote{\small{``Do what you fear, and the death of fear is certain.''\\-- Anthony Robbins}}

\nopagenumbers{}

\begin{document}

\maketitle

\section{教育背景}
\cventry{2010--2014}{学士}{计算机科学与技术}{上海交通大学}{}{}
%{
%\begin{itemize}
%\item \textbf{专业}: 计算机科学与技术
%\item \textbf{排名}: 21/122
%\item \textbf{部分专业课程/分}: 算法与复杂性(96), 离散数学(98), 操作系统(93), 编译原理(90), 计算机体系结构(91), 计算机组成实验(92), C++程序设计(91), 线性代数(94), 人工智能(93), 计算机网络(92), 数据库(90) and so on.\\
%\end{itemize}
%}

\cventry{2014--2017}{硕士}{计算机科学与技术}{上海交通大学}{}{}
%{
%\begin{itemize}
%\item \textbf{专业}: 计算机科学与技术
%\end{itemize}
%}

%\section{Master thesis}
%\cvline{title}{\emph{Title}}
%\cvline{supervisors}{Supervisors}
%\cvline{description}{\small Short thesis abstract}

\section{社区}
\cventry{博客}{\url{http://www.lifeofzjs.com}}{}{}{}{}
\cventry{github}{\url{https://www.github.com/zyearn}}{}{}{}{}


\section{工作经历}

\cventry{2018.8--至今}{上海哔哩哔哩科技有限公司}{高级研发工程师}{}{}
{
平台架构部
}
\vspace*{0.4\baselineskip}

\cventry{2017.2--2018.8}{百度(中国)有限公司}{高级研发工程师}{}{}
{
参与直播CDN相关和baidu-rpc的程序设计和开发工作。
}
\vspace*{0.4\baselineskip}

\cventry{2016.6--2016.9}{上海七牛信息技术有限公司}{实习}{}{}
{
参与核心产品线融合CDN的服务器端程序设计和开发工作。
}
\vspace*{0.4\baselineskip}

\cventry{2015.6--2015.9}{腾讯科技(深圳)有限公司}{实习}{}{}
{
完成了一套小型的游戏服务器系统,其中涉及到玩家登陆、简单地交互,玩家登出以及数据持久化等过程;学习并实现游戏服务器基本的架构。
}
\vspace*{0.4\baselineskip}

%\cventry{2014.7--2014.11}{上海仪电控股(集团)公司}{}{}{}
%{
%\begin{itemize}
%\item 职位: Web前后端 实习工程师\\
%参与项目网站的前端和后端开发。前端涉及页面开发,后端涉及NodeJS的开发、接口设计等。
%\end{itemize}
%}
%\vspace*{0.4\baselineskip}

\cventry{2013.8--2014.2}{上海依图科技网络有限公司}{实习}{}{}
{
负责部分后台进程架构的设计和程序的实现。这是我第一份实习,学到了很多比专业技能更宝贵的事情,比如一些做事的标准、流程、如何沟通,合作等。
}
%\vspace*{0.4\baselineskip}

%\section{论文发表}
%\cventry{2016}{Jiashun Zhu, Sumin Li, Linpeng Huang. Wamalloc: An Efficient Wear-Aware Allocator for Non-Volatile Memory. ICPADS 2016}{}{}{}{}

\section{核心项目经历}
%% \subsection{Research Projects}

\cventry{2017.2--至今}
{百度云直播}
{Live Streaming Service}
{media-server}
{流媒体服务器}
{
百度云音视频直播服务LSS,提供便捷接入的一站式端到端视频直播云服务,包含主播端视频采集推流SDK,云端直播流处理及全网分发和观众端播放器SDK,帮助企业客户快速搭建直播应用。
\begin{itemize}
    \item 流媒体服务器media-server负责人,主要负责业务需求实现、关键性能指标优化以及各种测试和监控的完善
    \item 完成media-server从开发到上线并逐渐替换现有模块的过程
    \item 负责CDN调度算法的抽象、设计、实现和推广
\end{itemize}
}
\vspace*{0.4\baselineskip}

\cventry{2017.2--至今}
{brpc}
{编程框架}
{}
{}
{
brpc目前已经是百度内部最主要的rpc框架,提供优秀的延时、吞吐,功能丰富,可扩展性高,有完备的调试/运维接口。
\begin{itemize}
    \item 项目地址:\url{https://github.com/brpc/brpc}
    \item 参与brpc的开源工作以及文档整理/翻译
    \item 参与brpc的核心功能开发,例如cmake编译系统支持,MacOS支持,HTTP/2实现等
    \item brpc的持续优化和推广,例如RTMP协议相关优化、性能优化等
\end{itemize}
}
\vspace*{0.4\baselineskip}

\cventry{2015.9--2017.6}
{Wamalloc}
{An Efficient Wear-Aware Allocator for Non-Volatile Memory}
{ICPADS2016}
{}
{
非易失性内存(NVM)由于其特殊的性质将会在未来的计算机系统中替代DRAM。然而,写损耗问题限制了它的实际应用,在这篇论文中,我提出Wamalloc,一种高性能的内存分配器,用于在软件层面延长NVM的使用寿命,同时也具备主流分配器的性能。
\begin{itemize}
    \item 论文地址: \url{https://ieeexplore.ieee.org/abstract/document/7823803/}
\end{itemize}
}
\vspace*{0.4\baselineskip}


\cventry{2015.9--2017.9}
{Zaver}
{yet another fast and efficient HTTP server}
{开源项目}
{}
{
Zaver的是一个基于事件驱动的高性能服务器,具备服务器的基本功能,编程模型是事件驱动 + 非阻塞I/O + 线程池。Zaver的特点是用非常少的代码展示了高性能服务器的核心结构,即可作为一个轻量级服务器使用,也可为开发者进一步学习网络编程打下基础。
\begin{itemize}
    \item 项目地址: \url{https://github.com/zyearn/zaver}
\end{itemize}
%Zaver的目的是帮助开发者理解基于epoll的高性能服务器是如何开发的。Nginx是一个非常好的服务器开发学习范例,但是它的规模之大让许多人望而却步。Zaver用非常少的代码展示了像Nginx这类高性能服务器的核心结构,为开发者进一步学习网络编程打下基础。
}
\vspace*{0.4\baselineskip}

\cventry{2015.12--2016.3}
{JOS}
{一个操作系统内核}
{课程项目}
{}
{
该项目是MIT操作系统课程6.828的大作业,一步步实现一个麻雀Exokernel内核。我主要实现了内存管理、进程、异常/中断、系统调用、页中断、多核支持、调度器、COWfork、抢占式内核、IPC、文件系统、网络支持等内核基本功能。
\begin{itemize}
    \item 项目地址: \url{https://github.com/zyearn/6.828-labs}
\end{itemize}
}
\vspace*{0.4\baselineskip}

%\cventry{2015.6-2015.9}
%{小型游戏服务器端设计与实现}
%{腾讯实习项目}
%{}
%{}
%{
%该项目是腾讯专为实习生设计的小型游戏服务器端开发训练项目。在此项目中,我实现了一个简易的分布式后端系统,接受玩家的登录,简单的交互,以及登出的整个流程,难点是如何设计一个高可靠、可容灾的系统;开发中用到的组件是腾讯内部的开发组件。
%}
%\vspace*{0.4\baselineskip}

%\cventry{2014.10-2014.12}
%{CE在线}
%{HTML, JavaScript, NodeJS, Express, Mysql}
%{外包项目}
%{}
%{
%\begin{itemize}
%    \item 项目地址: \url{http://ece114.com}
%\end{itemize}
%为某创业团队写的一个门户网站。前端涉及页面设计和接口设计;后端用了NodeJS的Express框架,NodeJS对I/O密集型任务有非常好的支持;用Mysql作为数据库。我的工作是部分前端开发和后端开发。
%}
%\vspace*{0.4\baselineskip}

%\cventry{2013.8--2013.12}
%{车牌智能识别项目}
%{C++}
%{company Projoct}
%{}
%{
%这个项目主要做车牌的智能识别。我负责工程部分(非机器识别算法)的一部分开发,以及工作进程间是如何组织和架构,涉及Linux下C++多线程开发。\\
%}
%\cventry{2012.10--2013.1}

%{单词终结者}
%{Html, Javascript}
%{课程设计}
%{Score:95/100}
%{
%越来越多的人疲倦于传统的背单词方式,比如说单词书或者电脑软件。
%单词终终结者是一款致力于将乐趣溶于单词学习中的网页游戏。将学习单词和游戏相结合是一个很好的创意。\\
%}
%\vspace*{0.4\baselineskip}

%\cventry{2012.9--2012.12}
%{用安卓程序和蓝牙控制小车}
%{Java}
%{课程设计}
%{Score:95/100}
%{
%这个项目致力于:\\
%1) 用JAVA写安卓程序\\
%2) 用蓝牙API来实现无线传输\\
%3) 写单片机程序来接受从手机传来的信号\\
%4) 根据接受的信号,驱动小车运动\\
%}
%\vspace*{0.4\baselineskip}

%\cventry{2012.10-2012.12}
%{SmallC编译器}
%{C, Flex, Yacc, Recursive-Descent Parser}
%{课程设计}
%{}
%{
%\begin{itemize}
%    \item 项目地址: \url{https://github.com/zyearn/SmallC}
%\end{itemize}
%该项目是为一个C子集语言而实现的一个编译器。用Flex做了词法分析,用Yacc做语法分析,用递归下降法做了机器码(x86)的生成。\\
%}

%\cventry{2012.4--2012.6}
%{文件系统设计与实现}
%{C, linux}
%{课程设计}
%{Score:93/100}
%{
%用C实现一个文件系统。真实硬盘用来模拟虚拟硬盘。设计原理是运用Inode,我的主要任务是设计和实现一个简易文件系统,支持一般操作比如ls,cd和编辑文件。\\
%}


%\vspace*{0.4\baselineskip}
%\cventry{2011.10--2012.2}
%{图形计算器}
%{Python}
%{课程设计}
%{Score:28/30}
%{
%这个项目致力于开发一个强大且易用的计算器。我们支持基础运算,画图和微积分(积分、求导)。我们用图形库实现了用户界面。\\
%这里是demo:\url{http://blog.csdn.net/zyearn/article/details/7282952}
%}

\section{奖项}
\cventry{2013--2016}{上海交通大学奖学金}{}{}{}{}
\cventry{2013}{第一届上海交大CCF编程马拉松 一等奖}{}{}{}{}
\cventry{2011}{上海交通大学奖学金}{}{}{}{}

\section{技能}
\cventry{语言/平台}{C, C++, Go, NodeJs/JavaScript, SQL, Python, Shell}{}{}{}{}
\cventry{文档}{LaTex, Vim}{}{}{}{}
\cventry{工具}{Git, MacOSX/Linux}{}{}{}{}
\cventry{基础}{扎实的Computer Science知识}{}{}{}{}
\cventry{英语}{CET6}{}{}{}{}

%\cventry{Games}{Warcraft 3}{}{}{}{}
% \cvline{Photography}{\small Digital photography is my newest hobby.}

%\closesection{}                   % needed to renewcommands
%\renewcommand{\listitemsymbol}{-} % change the symbol for lists

%\section{自我评价}
%\cventry{}{TODO}{}{}{}{}
\end{document}
