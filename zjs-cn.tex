\documentclass[11pt,a4paper]{moderncv}

% moderncv themes
%\moderncvtheme[blue]{casual}                 % optional argument are 'blue' (default), 'orange', 'red', 'green', 'grey' and 'roman' (for roman fonts, instead of sans serif fonts)
\moderncvtheme[blue]{classic}                % idem
\usepackage{xunicode, xltxtra}
\XeTeXlinebreaklocale "zh"
\widowpenalty=10000

%\setmainfont[Mapping=tex-text]{文泉驿正黑}
\setmainfont[BoldFont=黑体]{宋体}

% character encoding
%\usepackage[utf8]{inputenc}                   % replace by the encoding you are using
\usepackage{CJKutf8}

% adjust the page margins
\usepackage[scale=0.8]{geometry}
\recomputelengths                             % required when changes are made to page layout lengths
\setmainfont[Mapping=tex-text]{STFangsong}
\setsansfont[Mapping=tex-text]{STFangsong}
\CJKtilde

% personal data
\firstname{朱佳顺}
\familyname{}
\title{简历}               % optional, remove the line if not wanted

\mobile{+86 13764325590}                    % optional, remove the line if not wanted
\email{zhujiashun2010@gmail.com}                      % optional, remove the line if not wanted
%% \quote{\small{``Do what you fear, and the death of fear is certain.''\\-- Anthony Robbins}}

\nopagenumbers{}

\begin{document}

\maketitle

\section{教育背景}
\cventry{2010--2014}{本科生}{上海交通大学}{}{}
{
\begin{itemize}
\item \textbf{专业}: 计算机科学与技术
%\item \textbf{部分专业课程/分}: 算法与复杂性(96), 离散数学(98), 操作系统(93), 编译原理(90), 计算机体系结构(91), 计算机组成实验(92), C++程序设计(91), 线性代数(94), 人工智能(93), 计算机网络(92), 数据库(90) and so on.\\
\end{itemize}
}


\cventry{2014--至今}{硕士}{上海交通大学}{将于2017.03毕业}{}
{
\begin{itemize}
\item \textbf{专业}: 计算机科学与技术
\end{itemize}
}


%\section{Master thesis}
%\cvline{title}{\emph{Title}}
%\cvline{supervisors}{Supervisors}
%\cvline{description}{\small Short thesis abstract}


\section{社区}
\cventry{博客}{\url{http://www.lifeofzjs.com}}{}{}{}{}
\cventry{github}{\url{https://www.github.com/zyearn}}{}{}{}{}

\section{学术经历}
\cventry{2013.9--至今}{研究助理}{可靠分布式系统实验室}{}{}
{
\begin{itemize}
%\item 导师: \href{http://www.cs.sjtu.edu.cn/~fwu/index.html}{吴帆}(fwu@sjtu.edu.cn)
\item \textbf{研究方向}: 内存计算、操作系统相关
\end{itemize}
}

\section{实习经历}
\cventry{2014.6--2014.12}{上海仪电控股(集团)公司}{}{}{}
{
\begin{itemize}
\item 职位: 前端/后端工程师\\
我主要负责项目网站的部分前端和部分后端开发,前端用javascript/html/css,后端用Nodejs。
\end{itemize}
}

\cventry{2013.8--2014.2}{上海依图科技网络有限公司}{}{}{}
{
\begin{itemize}
\item 职位: Linux C++ 软件工程师
我主要负责部分架构的设计和程序的实现。这是一个创业公司,我学到了很多比专业技能更宝贵的事情,比如一些做事的标准,怎么合作等等。
\end{itemize}
}


\section{项目经历}
%% \subsection{Research Projects}
\cventry{2014.10-2014.12}
{CE在线}
{html, javascript, nodejs, express, mysql}
{outsourcing Project}
{}
{
这个项目我们为一个创业团队写了一个元器件领域中的门户网站。后端我们用了NodeJs的Express框架,它具有易开发,中间件功能强大的特点。数据库用了mysql。我的工作是前端开发,后端开发和数据库开发。\\
}

\vspace*{0.2\baselineskip}
\cventry{2014.6--2014.10}
{智能照明系统}
{php, javascrit, nodejs, socket.io}
{company Project}
{}
{
这个项目致力于让灯具的操作更加方便和人性化,比如在网页上调整灯具的亮度、色温、开关等。我的主要工作是开发后台网页(php),实时推送(nodejs)和前端开发(js)。\\
}

\vspace*{0.2\baselineskip}
\cventry{2013.12--2014.2}
{为公司对外服务设计API和实现sdk}
{C++}
{company Projoct}
{}
{
顾客需要一个简单易用的sdk来接入我们对外提供服务。我的主要工作是设计API和用C++实现sdk。难点在于API的简单清晰,实现的健壮性,处理不同的可能错误。\\
}

\vspace*{0.2\baselineskip}
\cventry{2013.8--2013.12}
{车牌智能识别项目}
{C++}
{company Projoct}
{}
{
这个项目主要做车牌的智能识别。我负责工程部分(非机器识别算法)的一部分开发,以及工作进程间是如何组织和架构,涉及Linux下C++多线程开发。\\
}

%\cventry{2012.10--2013.1}
%{单词终结者}
%{Html, Javascript}
%{课程设计}
%{Score:95/100}
%{
%越来越多的人疲倦于传统的背单词方式,比如说单词书或者电脑软件。
%单词终终结者是一款致力于将乐趣溶于单词学习中的网页游戏。将学习单词和游戏相结合是一个很好的创意。\\
%}

%\vspace*{0.2\baselineskip}
%\cventry{2012.9--2012.12}
%{用安卓程序和蓝牙控制小车}
%{Java}
%{课程设计}
%{Score:95/100}
%{
%这个项目致力于:\\
%1) 用JAVA写安卓程序\\
%2) 用蓝牙API来实现无线传输\\
%3) 写单片机程序来接受从手机传来的信号\\
%4) 根据接受的信号,驱动小车运动\\
%}

\vspace*{0.2\baselineskip}
\cventry{2012.10-2012.12}
{实现SmallC的编译器}
{C, Flex, Bison}
{课程设计}
{Score:93/100}
{
这个项目包含了高级语言如何编译成汇编语言的全部过程。我担任整个项目的设计者和实现者,其中包括了词法分析,语法分析,语义分析,中间代码生成,优化和最终代码生成。\\
}

\vspace*{0.2\baselineskip}
\cventry{2012.4--2012.6}
{为操作系统实现文件系统}
{C, linux}
{课程设计}
{Score:93/100}
{
用C实现一个文件系统。真实硬盘用来模拟虚拟硬盘。设计原理是运用Inode,我的主要任务是设计和实现一个简易文件系统,支持一般操作比如ls,cd和编辑文件。\\
}


\vspace*{0.2\baselineskip}
\cventry{2011.10--2012.2}
{图形计算器}
{Python}
{课程设计}
{Score:28/30}
{
这个项目致力于开发一个强大且易用的计算器。我们支持基础运算,画图和微积分(积分、求导)。我们用图形库实现了用户界面。\\
这里是demo:\url{http://blog.csdn.net/zyearn/article/details/7282952}
}

\section{奖项}
\cventry{2013}{上海交通大学奖学金}{}{}{}{}
\cventry{2013}{第一届上海交大CCF编程马拉松 一等奖}{}{}{}{}
\cventry{2011}{上海交通大学奖学金}{}{}{}{}
\cventry{2011}{上海交通大学校团委优秀干事、主任}{}{}{}{}
\cventry{2010}{上海市第四届物理竞赛 三等奖}{}{}{}{}

\section{英语能力}
\cventry{CET 6}{559}{}{}{}{}


\section{技能}
\cventry{Language/Platform}{C, C++, NodeJs/Javascript, Python, Shell}{}{}{}{}
\cventry{Document}{LaTex, Vim}{}{}{}{}
\cventry{Tools}{Git, MacOSX/Linux}{}{}{}{}
%\cventry{Games}{Warcraft 3}{}{}{}{}
% \cvline{Photography}{\small Digital photography is my newest hobby.}

%\closesection{}                   % needed to renewcommands
%\renewcommand{\listitemsymbol}{-} % change the symbol for lists

\end{document}
